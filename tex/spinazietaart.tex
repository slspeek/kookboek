\begin{recipe}
[ %
    preparationtime = {\unit[60]{min}},
    bakingtime = {\unit[35]{min} oventijd},
    bakingtemperature = {\protect\bakingtemperature{
	    fanoven=\unit[200]{\textcelcius}}},
    portion = {\portion{4}},
    source = {\href{https://www.ah.nl/allerhande/recept/R-R177680/spinazietaart}{Spinazietaart recept - Allerhande | Albert Heijn}}
]
{Spinazietaart}

    \ingredients
    {%
        \unit[450]{g} & diepvries spinazie \`{a} la cr\`{e}me \\
        \unit[400]{g} & rundergehakt \\
        4 & eieren \\
        1 & ui \\
        \unit[4]{teentjes} & knoflook \\ 
        \unit[150]{g} & geraspte kaas \\
        \unit[2]{el} & paneermeel \\
        & Worcestersaus \\
        & Italiaanse keukenkruiden \\
        & Olijfolie \\
        & zout \\
        & peper 
    }

    \preparation
    {%
        \step Ontdooi de spinazie (in een pannetje op laag vuur of in de magnetron).
        En laat haar zeker een halfuur uitlekken in een zeef.
        \step Begin met het maken van de \hyperref[rec:taartbodem]{taartbodem}.
        \step Snijd de knoflook fijn en snipper de ui.
        \step Bak het gehakt rul. Voeg de knoflook en ui toe. Breng op smaak met peper, zout 
        en Worcestersaus. Bak het wat langer door om het vocht te laten verdampen.
        \step Rasp de kaas.
        \step Meng de eieren en de kaas in een ruime kom. Laat de gehakt afkoelen voor je
        het toevoegd. Roer de uitgelekte spinazie erdoor.
        \step Bestrooi de voorgebakken taartbodem met 2 eetlepels paneermeel.
        \step Doe de vulling in de taartbodem en bak het geheel in 35 minuten op \unit[200]{\textcelcius}.
        gaar.

    }
    
\end{recipe}
\label{rec:spinazietaart}
