\begin{recipe}
[ %
    preparationtime = {\unit[45]{min}},
    bakingtime = {\unit[40]{min} oventijd},
    bakingtemperature = {\protect\bakingtemperature{
	    fanoven=\unit[200]{\textcelcius}}},
    portion = {\portion{4}},
    calory,
    source = {\href{https://www.ah.nl/allerhande/recept/R-R665332/broccoliquiche}{Albert Heijn}}
]
{Broccolitaart}

    \ingredients
    {%
	\unit[1]{pakje} & bladerdeeg\\
	\unit[1]{el} & boter\\
	\unit[1]{el} & paneermeel\\
	\unit[300]{g} & broccoli\\
	3 & eieren\\
	\unit[4]{el} & slagroom\\
	\unit[10]{sprietjes} & bieslook\\
	\unit[100]{g} & gezouten cashewnoten \\
	\unit[250]{g} & camembert\\
        & zout \\
        & peper \\
    }

    \preparation
    {%
	    \step Ontdooi de bladerdeeg.
	    \step Vet de springvorm in met boter.
	    \step Bekleed de springvorm met bladerdeeg. Zorg dat het goed aansluit. Bestrooi de bodem met paneermeel.
	    \step Was de broccoli. Snijd deze in roosjes en kleine plakjes. Kook deze ongeveer 5 minuten. Spoel ze onder koud water af en laat ze in een vergiet uitlekken.
	    \step Klop de eieren met de slagroom los in een kommetje. Voeg peper, zout en de bieslook fijngeknipt toe.
	    \step Kraak de cashnoten met de onderkant van een mok op de snijplank.
	    \step Snijd de camembert in plakken.
	    \step Verwarm de oven voor op 200\textcelcius
	    \step Rangschik de broccoli op de taartbodem en druk deze een beetje aan met een vork.
	    \step Strooi de gekraakte noten erover.
	    \step Leg de camembert erover.
	    \step Schenk het slagroommengsel erover.
	    \step Plaats de springvorm op een rooster onder het midden van de oven. Laat de taart gaar bakken in 40 minuten.
	    \step Neem de taart uit de oven en laat hem 10 minuten afkoelen. Verwijder voorzichtig de springvorm.

    }
    
\end{recipe}
