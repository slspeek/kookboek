\begin{recipe}
[ %
    preparationtime = {\unit[\nicefrac{3}{4}]{h}},
    bakingtime = {\unit[20]{min} koken},
    portion = {\portion{2}},
    calory,
    source = {\href{https://www.rudolphskitchen.nl/recipe/vegetarische-stamppot-rauwe-andijvie-met-kaas-en-noten/}{Rudolph's kitchen}}
]
{Andijviestamppot met noten}

    \ingredients
    {%
    \unit[600]{g} & aardappelen\\
	\unit[500]{g} & fijngesneden andijvie \\
	\unit[1]{blokje} & groenteboullion \\
  \unit[1]{el} & mosterd \\
	\unit[150]{ml} & melk \\
	\unit[15]{g} & boter \\
	\unit[200]{g} & Goudse kaas in blokjes \\
	\unit[125]{g} & cashewnoten of pecannoten \\
    }

    \preparation
    {%
    \step Schil de aardappelen. Kook ze in water waar ze net onder staan met een boullionblokje aan de kook. Giet ze na 20 minuten af. En zet de pan heel even terug op het vuur om de aardappelen droog te stomen.
    \step Stamp de aardappelen fijn. Verwarm de melk en roer deze samen met het klontje boter door de aardappelen. 
	\step Roer de andijvie met een spatel door de aardappelpuree en de pan warm op heel laag vuur.
	\step Roer de kaasblokjes en de mosterd door het mengsel. Stop op tijd met roeren zodat de kaasblokjes nog zichtbaar blijven in de stamppot.
	\step Zet het deksel op de pan en houdt de stamppot enkele minuten warm zodat de kaas (gedeeltelijk) kan smelten. Rooster intussen de noten in een koekepan.
	\step Verdeel de stamppot over twee borden en strooi de noten eroverheen.
    }
    
\end{recipe}
