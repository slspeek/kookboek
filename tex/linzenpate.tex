\begin{recipe}
[ %
    preparationtime = {\unit[1 1/2]{h}},
    bakingtime = {\unit[75]{m}},
    bakingtemperature={\protect\bakingtemperature{
        fanoven=\unit[175]{\textcelcius},
        gasstove=Level 2}},
    portion = {\portion{4}},
    calory,
    source = {Lekker gezond vegetarisch}
]
{Linzenpat\'{e}}

    % \introduction
    % {
    %   Hoi
    % }

    \ingredients
    {%
        \unit[800]{g} & linzen\\
        6 & zongedroogde tomaatjes op olie\\
        6 blaadjes  & basilicum\\
        3  & eidooiers\\
        \unit[1]{dl}  & yoghurt\\
        \unit[2]{el}  & groene pesto\\
        3  & lente-uitjes in ringetjes \\
        \unit[75]{g} & zwarte olijven zonder pit in plakjes \\
        \unit[1]{el} & Proven\c{c}aalse kruiden \\
         & zout, peper \\
         \unit[200]{g} & rijst \\
    }

    \preparation
    {%
        \step Pureer de helft van de linzen, van de tomaatjes en van de basilicum
        met de eidooiers. yoghurt en pesto in de keukenmachine.
        \step Snijd de resterende tomaatjes en basilicum in stukjes en schep deze samen met
        de resterende linzen, de lente-uitjes, olijven en Proven\c{c}aalse kruiden
        door de linzenpuree. Breng op smaak met peper en zout.
        \step Schep de puree in een ingevet cakeblik. Zet het cakeblik in een met
        heet water gevulde grote schaal 75 minuten in de oven.
        \step Zet de rijst op. Als ze klaar is en giet je de rijst af en laat haar staan met de deksel op de pan.
        \step Laat de pat\'{e} 10 minuten in de vorm afkoelen en stort hem eruit.
        Snijd hem in plakken.
      }

      \suggestion
      {
        Lekker met rucola en tomaten salade.
      }

\end{recipe}
