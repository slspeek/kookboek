\begin{recipe}
[ %
    preparationtime = {\unit[10]{min} (totaal \unit[60]{min})},
     bakingtime = {\unit[20]{min} oventijd},
    bakingtemperature = {\protect\bakingtemperature{
	    fanoven=\unit[180]{\textcelcius}}},
    portion = {\portion{4}},
    source = {\href{https://www.laurasbakery.nl/basisdeeg-voor-quiche-en-hartige-taart/}{Basisrecept: quichedeeg en hartige taartdeeg - Laura's Bakery}}
]
{Hartige taartbodem}

    \ingredients
    {%
	  \unit[300]{g} & bloem\\
      \unit[2-3]{el} & water \\
      \unit[150]{g} & ongezouten roomboter \\
      1 & ei \\
	    & zout \\
    }

    \preparation
    {%
	    \step Meng de bloem, het ei en de boter in een ruime kom. Voeg een snufje zout toe.
        Begin met \unit[2]{el} water. Als het te droog (kruimelig) blijft, voeg je de
        derde eetlepel water toe. Kneed het totdat het een egale structuur heeft.
	    \step Doe het deeg in een plastic zakje en leg het een half uur in de koelkast.
        \step Vet je quichevorm in met boter.
        \step Verwarm de oven voor op \unit[180]{\textcelcius}.
        \step Rol het deeg uit op een oppervlak groter dan je bakvorm. Bestroei het oppervlak
        van te voren met bloem.
        \step Rol het deeg om de deegroller en breng het over in de quichevorm.
        \step Prik met een vork veel gaatjes in de bodem.
        \step Verkeukel een blad bakpapier en doe daarin de steunvulling.
        \step Blind bak de bodem 20 minuten in de voorverwarmde oven.
    }
    
    \suggestion
    {
        Terug naar \hyperref[rec:spinazietaart]{spinazietaart}.
    }
    
\end{recipe}
\label{rec:taartbodem}
