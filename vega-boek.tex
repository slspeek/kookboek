\documentclass[%
a4paper,
%twoside,
12pt
]{article}

% encoding, font, language
\usepackage[T1]{fontenc}
\usepackage[latin1]{inputenc}
\usepackage{lmodern}
\usepackage[dutch]{babel}

\usepackage{nicefrac}

\usepackage[
    nowarnings,
    %myconfig
]
{xcookybooky}

\definecolor{mygreen}{rgb}{0,.2,0}
\DeclareRobustCommand{\textcelcius}{\ensuremath{^{\circ}\mathrm{C}}}

\setRecipeColors
{%
    recipename = mygreen,
    ing = black,
    inghead = black,
    prep,
    prephead,
    hint,
    hinthead,
    numeration = mygreen,
}

\setcounter{secnumdepth}{1}
\renewcommand*{\recipesection}[2][]
{%
    \subsection[#1]{#2}
}
\renewcommand{\subsectionmark}[1]
{% no implementation to display the section name instead
}


%%%%%%%%%%
% hyperref
\usepackage{hyperref}    % must be the last package
\hypersetup{%
    pdfauthor            = {Steven Speek},
    pdftitle             = {Vegetarisch kookboek},
    pdfsubject           = {recepten},
    pdfstartview         = {FitV},
    pdfview              = {FitH},
    pdfpagemode          = {UseNone}, % Options; UseNone, UseOutlines
    bookmarksopen        = {true},
    pdfpagetransition    = {Glitter},
    colorlinks           = {true},
    linkcolor            = {black},
    urlcolor             = {black}
    citecolor            = {black},
    filecolor            = {black},
}
% hyperref
%%%%%%%%%%



\begin{document}

\title{Vegetarisch kookboek}
\author{Steven Speek}
\maketitle

\tableofcontents

\setHeadlines
{% translation
    inghead = Ingredi\"{e}nten,
    prephead = Bereiding,
    hinthead = Tip,
    continuationhead = Voortzetting,
    continuationfoot = Gaat verder op volgende bladzijde,
    portionvalue = personen,
}

\begin{recipe}
[ %
    preparationtime = {\unit[\nicefrac{3}{4}]{h}},
    bakingtime = {\unit[20]{min} koken, \unit[10]{min} grillen},
    portion = {\portion{2}},
    calory,
    source = {Zusanova}
]
{Vegetarische hutspot}

    \ingredients
    {%
        \unit[500]{g} & aardappelen\\
        \unit[500]{g} & winterpeen of wortelen \\
        2 & grote uien \\
        \unit[1.5]{el} & gembersiroop \\
        \unit[0.5]{el} & sambal \\
        \unit[200]{g} & geitenkaas in plakken \\
        half bosje & koriander \\
        \unit[1]{el} & plantaardige olie \\
         & zout \\
    }

    \preparation
    {%
        \step Schil de aardappelen en de wortelen snijd ze in gelijke stukken.
        De uien in ringen snijden. Hak de koriander fijn.
        \step Kook de aardappelen met de wortel en zonder de ui in 20 minuten gaar. Doe er een beetje zout bij.
        \step Fruit de uiringen in de olie.
        \step Giet de aardappelen met de wortelen af. Stamp de ui, de sambal en gembersiroop door het aardappel mengsel.
        \step Schep dit mengsel in de een ovenschaal. Verdeel de plakken geitenkaas over de hutspot.
        Druppel nog wat gembersiroop op de geitenkaas.
        Grill dit net zolang tot de geitenkaas een licht bruine gloed krijgt.
        \step Serveer de schotel met de gehakte koriander.
    }
    
\end{recipe}

\begin{recipe}
[ %
    preparationtime = {\unit[1]{h}},
    bakingtime,
    portion = {\portion{2}},
    calory,
    source = {Studentenkookboek, Berty van Essen}
]
{Tahoe met ketjap}

    \ingredients
    {%
        klein blok & tahoe (biologisch) \\
        \unit[200]{g} & zilvervlies rijst\\
        $$1/2$$ & citroen \\
        \unit{2}{el} & olie \\
        een kleine  &  ui \\
        3 teentjes  & knoflook \\
        \unit{2}{tl} & sambal \\
        \unit{3}{el} & ketjap \\
        \unit{200}{g}  & taug\'{e} \\
        & zout \\
    }

    \preparation
    {%
        \step Snijd de tahoe in blokjes of balkjes van 1 cm. Wrijf ze in met wat zout.
        Sprenkel de citroen erover en laat dit minstens 10 minuten staan.
        \step Snipper de ui en de knoflook.
        \step Verhit de olie in een koekepan en bak de tahoe snel bruin. Neem ze uit de
        pan en laat ze in een vergiet uitlekken.
        \step Bak in de zelfde olie de ui en knoflook glazig en goudgeel. Voeg
        de sambal, ketjap en zout toe. Was de taug\'{e} en bak dat even mee. Voeg eventueel
        wat water toe als de saus te dik wordt. Voeg de tahoe weer toe en je kunt opdienen.
    }

    \suggestion
    {
      Hier past goed Roedjak bij. Met iets zoets als toetje is het dan een hele maaltijd.
    }
\end{recipe}

\begin{recipe}
[ %
    preparationtime = {\unit[\nicefrac{1}{2}]{h}},
    bakingtime,
    portion = {\portion{2}},
    calory,
    source = {Studentenkookboek, Berty van Essen}
]
{Roedjak}

    \ingredients
    {%
        \unit{1}{tl} & sambal \\
        \unit{1}{el} & bruine suiker \\
        \nicefrac{1}{2} & citroen \\
        \unit{\nicefrac{1}{2}}{el} & ketjap \\
         1 & handappel \\
         1 & handpeer \\
         \nicefrac{1}{2} & komkommer \\
         1 blikje & ananas \\
         & zout \\
    }

    \preparation
    {%
        \step Maak in een kom een sausje van de sambal, de suiker, het citroensap,
        de ketjap en zout naar smaak.
        \step Schil de appel en de peer, verwijder de klokhuizen en snijd ze in blokjes.
        Voeg de saus toe en schep het meteen door elkaar.
        \step Snijd de komkommer in blokjes, schep deze met uitgelekt blikje
        ananas door de rest van het fruit.
    }

\end{recipe}
\label{rec:roedjak}



\end{document}
