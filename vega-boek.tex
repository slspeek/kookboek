\documentclass[%
a4paper,
%twoside,
12pt
]{article}

% encoding, font, language
\usepackage[T1]{fontenc}
\usepackage[latin1]{inputenc}
\usepackage{lmodern}
\usepackage[dutch]{babel}

\usepackage{nicefrac}

\usepackage[
    nowarnings,
    %myconfig
]
{xcookybooky}

\definecolor{mygreen}{rgb}{0,.2,0}
\DeclareRobustCommand{\textcelcius}{\ensuremath{^{\circ}\mathrm{C}}}

\setRecipeColors
{%
    recipename = mygreen,
    ing = black,
    inghead = black,
    prep,
    prephead,
    hint,
    hinthead,
    numeration = mygreen,
}

\setcounter{secnumdepth}{1}
\renewcommand*{\recipesection}[2][]
{%
    \subsection[#1]{#2}
}
\renewcommand{\subsectionmark}[1]
{% no implementation to display the section name instead
}


%%%%%%%%%%
% hyperref
\usepackage{hyperref}    % must be the last package
\hypersetup{%
    pdfauthor            = {Steven Speek},
    pdftitle             = {Vegetarisch kookboek},
    pdfsubject           = {recepten},
    pdfstartview         = {FitV},
    pdfview              = {FitH},
    pdfpagemode          = {UseNone}, % Options; UseNone, UseOutlines
    bookmarksopen        = {true},
    pdfpagetransition    = {Glitter},
    colorlinks           = {true},
    linkcolor            = {black},
    urlcolor             = {black}
    citecolor            = {black},
    filecolor            = {black},
}
% hyperref
%%%%%%%%%%



\begin{document}

\title{Vegetarisch kookboek}
\author{Steven Speek}
\maketitle

\tableofcontents

\setHeadlines
{% translation
    inghead = Ingredi\"{e}nten,
    prephead = Bereiding,
    hinthead = Tip,
    continuationhead = Voortzetting,
    continuationfoot = Gaat verder op volgende bladzijde,
    portionvalue = personen,
}

\newpage

\begin{recipe}
[ %
    preparationtime = {\unit[$3/4$]{h}},
    bakingtime = {\unit[20]{min} koken, \unit[10]{min} grillen},
    portion = {\portion{2}},
    calory,
    source = \href{http://www.smulweb.nl/community/pagina/zusanova}{Zusanova}
]
{Vegetarische hutspot}

    \ingredients
    {%
        \unit[500]{g} & aardappelen\\
        \unit[500]{g} & winterpeen of wortelen \\
        2 & grote uien \\
        \unit[1.5]{el} & gembersiroop \\
        \unit[0.5]{el} & sambal \\
        \unit[200]{g} & geitenkaas in plakken \\
        half bosje & koriander \\
        \unit[1]{el} & plantaardige olie \\
         & zout \\
    }

    \preparation
    {%
        \step Schil de aardappelen en snij ze in gelijke stukken.
        Schil de wortelen en snij ze in even grote stukken als de aardappelen.
        De uien in ringen snijden. Hak de koriander fijn.
        \step Kook de aardappelen met de wortel en zonder de ui in 20 minuten gaar. Doe er een beetje zout bij.
        \step Fruit de uiringen in de olie.
        \step Giet de aardappelen met de wortelen af. Stamp de ui, de sambal en gembersiroop door het aardappel mengsel.
        \step Schep dit mengsel in de een ovenschaal. Verdeel de plakken geitenkaas over de hutspot.
        Druppel nog wat gembersiroop op de geitenkaas.
        Grill dit net zolang tot de geitenkaas een licht bruine gloed krijgt.
        \step Serveer de schotel met de gehakte koriander.
    }

\end{recipe}



\end{document}
