\documentclass[%
a4paper,
%twoside,
12pt
]{article}

% encoding, font, language
\usepackage[T1]{fontenc}
\usepackage[latin1]{inputenc}
\usepackage{lmodern}
\usepackage[dutch]{babel}

\usepackage{nicefrac}

\usepackage[
    nowarnings,
    %myconfig
]
{xcookybooky}

\definecolor{mygreen}{rgb}{0,.2,0}
\DeclareRobustCommand{\textcelcius}{\ensuremath{^{\circ}\mathrm{C}}}

\setRecipeColors
{%
    recipename = mygreen,
    ing = black,
    inghead = black,
    prep,
    prephead,
    hint,
    hinthead,
    numeration = mygreen,
}

\setcounter{secnumdepth}{1}
\renewcommand*{\recipesection}[2][]
{%
    \subsection[#1]{#2}
}
\renewcommand{\subsectionmark}[1]
{% no implementation to display the section name instead
}


%%%%%%%%%%
% hyperref
\usepackage{hyperref}    % must be the last package
\hypersetup{%
    pdfauthor            = {Steven Speek},
    pdftitle             = {Vegetarisch kookboek},
    pdfsubject           = {recepten},
    pdfstartview         = {FitV},
    pdfview              = {FitH},
    pdfpagemode          = {UseNone}, % Options; UseNone, UseOutlines
    bookmarksopen        = {true},
    pdfpagetransition    = {Glitter},
    colorlinks           = {true},
    linkcolor            = {black},
    urlcolor             = {black}
    citecolor            = {black},
    filecolor            = {black},
}
% hyperref
%%%%%%%%%%



\begin{document}

\title{Vegetarisch kookboek}
\author{Steven Speek}
\maketitle

\tableofcontents

\setHeadlines
{% translation
    inghead = Ingredi\"{e}nten,
    prephead = Bereiding,
    hinthead = Tip,
    continuationhead = Voortzetting,
    continuationfoot = Gaat verder op volgende bladzijde,
    portionvalue = personen,
}

\section{Hoofdgerechten}
% Deze gerechten heb ik minstens een keer gemaakt en lekker genoeg bevonden.
\begin{recipe}
[ %
    preparationtime = {\unit[1]{h}},
    bakingtime,
    portion = {\portion{2}},
    calory,
    source = {Studentenkookboek, Berty van Essen}
]
{Tahoe met ketjap}

    \ingredients
    {%
        klein blok & tahoe (biologisch) \\
        \unit[200]{g} & zilvervlies rijst\\
        $$1/2$$ & citroen \\
        \unit{2}{el} & olie \\
        een kleine  &  ui \\
        3 teentjes  & knoflook \\
        \unit{2}{tl} & sambal \\
        \unit{3}{el} & ketjap \\
        \unit{200}{g}  & taug\'{e} \\
        & zout \\
    }

    \preparation
    {%
        \step Snijd de tahoe in blokjes of balkjes van 1 cm. Wrijf ze in met wat zout.
        Sprenkel de citroen erover en laat dit minstens 10 minuten staan.
        \step Snipper de ui en de knoflook.
        \step Verhit de olie in een koekepan en bak de tahoe snel bruin. Neem ze uit de
        pan en laat ze in een vergiet uitlekken.
        \step Bak in de zelfde olie de ui en knoflook glazig en goudgeel. Voeg
        de sambal, ketjap en zout toe. Was de taug\'{e} en bak dat even mee. Voeg eventueel
        wat water toe als de saus te dik wordt. Voeg de tahoe weer toe en je kunt opdienen.
    }

    \suggestion
    {
      Hier past goed Roedjak bij. Met iets zoets als toetje is het dan een hele maaltijd.
    }
\end{recipe}

\begin{recipe}
[ %
    preparationtime = {\unit[\nicefrac{1}{2}]{h}},
    portion = {\portion{2}},
    calory,
]
{Tabbouleh}

    \introduction {
      Dit gerecht vraagt vanwege de avocado planning. Deze moet enkele dagen
      tot een week voor het maken worden aangeschaft.
    }
    \ingredients
    {%
        \unit[150]{ml} & couscous \\
        1 & komkommer \\
        2 & tomaten \\
        1 & avocado \\
        1 zakje & gesneden rauwkost \\
        1 bos & koriander \\
        2 tenen & knoflook \\
        \nicefrac{1}{2} beker & yoghurt\\
        \nicefrac{1}{2} & citroen\\
        & peper en zout \\
        & olijfolie \\
        zakje & shoarmakruiden;
    }

    \preparation
    {%
        \step Doe een scheut olijfolie bij de de couscous. Wel dit met 150 ml kokend water en laat haar vijf minuten staan.
        \step Snijd de tomaten en de helft van de komkommer in blokjes. Knip koriander fijn.
         Snijd de avocado klein. Meng dit alles door de couscous met peper en zout.
        \step Voor het sausje rasp je de overgebleven komkommer.
        Knijp de 2 tenen knoflook erdoorheen, yoghurt en citroen erbij en
        op smaak brengen met peper en zout.
        \step Serveer de tabbouleh, de komkommer-saus en de shoarmakruiden los van elkaar.
      }

\end{recipe}

\begin{recipe}
[ %
    preparationtime = {\unit[\nicefrac{3}{4}]{h}},
    bakingtime = {\unit[20]{min} koken, \unit[10]{min} grillen},
    portion = {\portion{2}},
    calory,
    source = {Zusanova}
]
{Vegetarische hutspot}

    \ingredients
    {%
        \unit[500]{g} & aardappelen\\
        \unit[500]{g} & winterpeen of wortelen \\
        2 & grote uien \\
        \unit[1.5]{el} & gembersiroop \\
        \unit[0.5]{el} & sambal \\
        \unit[200]{g} & geitenkaas in plakken \\
        half bosje & koriander \\
        \unit[1]{el} & plantaardige olie \\
         & zout \\
    }

    \preparation
    {%
        \step Schil de aardappelen en de wortelen snijd ze in gelijke stukken.
        De uien in ringen snijden. Hak de koriander fijn.
        \step Kook de aardappelen met de wortel en zonder de ui in 20 minuten gaar. Doe er een beetje zout bij.
        \step Fruit de uiringen in de olie.
        \step Giet de aardappelen met de wortelen af. Stamp de ui, de sambal en gembersiroop door het aardappel mengsel.
        \step Schep dit mengsel in de een ovenschaal. Verdeel de plakken geitenkaas over de hutspot.
        Druppel nog wat gembersiroop op de geitenkaas.
        Grill dit net zolang tot de geitenkaas een licht bruine gloed krijgt.
        \step Serveer de schotel met de gehakte koriander.
    }
    
\end{recipe}

\begin{recipe}
[ %
    preparationtime = {\unit[1 1/2]{h}},
    bakingtime = {\unit[75]{m}},
    bakingtemperature={\protect\bakingtemperature{
        fanoven=\unit[175]{\textcelcius},
        gasstove=Level 2}},
    portion = {\portion{4}},
    calory,
    source = {Lekker gezond vegetarisch}
]
{Linzenpat\'{e}}

    % \introduction
    % {
    %   Hoi
    % }

    \ingredients
    {%
        \unit[800]{g} & linzen\\
        6 & zongedroogde tomaatjes op olie\\
        6 blaadjes  & basilicum\\
        3  & eidooiers\\
        \unit[1]{dl}  & yoghurt\\
        \unit[2]{el}  & groene pesto\\
        3  & lente-uitjes in ringetjes \\
        \unit[75]{g} & zwarte olijven zonder pit in plakjes \\
        \unit[1]{el} & Proven\c{c}aalse kruiden \\
         & zout, peper \\
         \unit[200]{g} & rijst \\
    }

    \preparation
    {%
        \step Pureer de helft van de linzen, van de tomaatjes en van de basilicum
        met de eidooiers. yoghurt en pesto in de keukenmachine.
        \step Snijd de resterende tomaatjes en basilicum in stukjes en schep deze samen met
        de resterende linzen, de lente-uitjes, olijven en Proven\c{c}aalse kruiden
        door de linzenpuree. Breng op smaak met peper en zout.
        \step Schep de puree in een ingevet cakeblik. Zet het cakeblik in een met
        heet water gevulde grote schaal 75 minuten in de oven.
        \step Zet de rijst op. Als ze klaar is en giet je de rijst af en laat haar staan met de deksel op de pan.
        \step Laat de pat\'{e} 10 minuten in de vorm afkoelen en stort hem eruit.
        Snijd hem in plakken.
      }

      \suggestion
      {
        Lekker met \hyperref[rec:rucola_met_tomaat]{rucola en tomaten salade}.
      }

\end{recipe}

\begin{recipe}
[ %
    preparationtime = {\unit[1]{h}},
    portion = {\portion{4}},
    calory,
]
{Noten burgers met avocado-aioli}

    \ingredients
    {%
        1 & ui \\
        2 & tenen knoflook \\
        \unit[75]{g} & cashewnoten \\
        \unit[75]{g} & pecannoten \\
        \unit[50]{g} & kastanjechampignons \\
        \unit[1]{el} & ketchup \\
        \unit[1]{tl} & sambal oelek \\
        \unit[1]{el} & pindakaas \\
        1 & Stokbrood \\
        1 & ei \\
        \unit[70]{g} & kidneybonen \\
        \unit[80]{g} & havermout \\
        \unit[4]{el} & olijfolie \\
        1 rijpe & avocado \\
        1 & limoen \\
        \unit[2]{el} & mayonaise \\
        & zout en peper \\
    }

    \preparation
    {%
        \step Spoel de kidneybonen af in een zeef. Laat goed uitlekken.
        \step Pel en schil de knoflook en ui. Doe de ui, 1 teen knoflook,
        cashewnoten, pecannoten, kastanjechampignons, ketchup, sambal oelek,
        pindakaas, het ei, de kidneybonen en havermout in een keukenmachine.
        Voeg wat peper en zout toe.
        Maal kort met de pulseknop. Het is fijn als er nog structuur in de substantie zit.
        \step Verdeel het mengsel in 4 hamburgers en laat minimaal 1 uur in de koelkast opstijven.
        Voordat je de hamburgers barbecuet kun je ze het beste 6 minuten in 4 eetlepels olijfolie voorgaren in de pan.
        \step Halveer de avocado en verwijder de pit. Haal het vruchtvlees er met een lepel uit en leg in een diepe kom.
        Pers de limoen uit. Voeg de mayonaise, het limoensap en de andere teen knoflook toe aan het avocadovruchtvlees.
        Breng op smaak met zout en peper en pureer glad met een staafmixer.
        \step Gril de vegaburgers 2 minuten op de bbq of in de oven.
        Serveer met de avocado-aioli en wat brood.

      }

\end{recipe}

\begin{recipe}
[ %
    preparationtime = {\unit[1 1/2]{h}},
    portion = {\portion{2}},
    calory,
    source = {Gerbina van Hurk},
]
{Turkse bonen salade}

    \ingredients
    {%
         \unit[300]{g} & witte bonen \\
         1  & citroen \\
            & of   \\
         \unit[4]{el} & azijn \\
         1 teentje & knoflook \\
         1 & sjalotje \\
         \unit[3]{el} & fijngehakte tuinkruiden, zoals peterselie, munt, dille \\
         2 & eieren \\
         \unit[5]{el} & olijfolie \\
         paar & cherrytomaatjes \\
         & of \\
         1  & kleine paprika \\
         & zout en peper\\
    }

    \preparation
    {%
        \step Kook de eieren hard. Snipper het sjalotje.
        \step Pers de citroen uit en meng het sap met olijfolie, uitgepreste knoflook en zout.
        Meng de gesneden sjalotje door de dressing en schep de dressing door de (liefst nog lauwwarme) bonen.
        \step Voeg de paprika in stukjes gesneden toe, of snijd de cherrytomaatje doormidden
        voor je ze toevoegd.
        \step Meng de verse tuinkruiden door de bonensalade.
        Laat de bonensalade minimaal een uur staan alvorens deze op te dienen.
        Decoreer de salade met plakjes hard gekookt ei.
      }

\end{recipe}
\label{rec:turkse-bonen-salade}

\begin{recipe}
[ %
    preparationtime = {\unit[20]{m}},
    portion = {\portion{2}},
    calory,
    source = {MissRiz op Smulweb}
]
{Paksoi met walnoten}

    \ingredients
    {%
      1 struik & paksoi \\
      1 & vleestomaat \\
      \unit[1]{tl} & sambal \\
      2 teentjes & knoflook \\
      6 & walnoten \\
      1 & bouillonblokje \\
      snufje & djintan \\
      snufje & ketoembar\\
      \unit[1]{tl}  & sesamzaad
    }

    \preparation
    {%
        \step Paksoi in repen Paksoi in repen snijden,
              de stengels wat dunner dan de bladeren,
              knoflook fijn snijden,
              de tomaat in blokjes, walnoten kraken (niet hakken!)
        \step Knoflook 1 minuut fruiten, dan 2 minuten met de sambal erbij.
              tomaat erbij. Even bakken. paksoi erbij, bouillonblokje erboven
              verkruimelen, 2-3 minuten roerbakken. Vocht afgieten in zeef.
              Terug in de pan, en de kruiden erbij doen.
              Als laatste de walnoten erdoor scheppen.
              Niet te lang in de pan laten, want de paksoi moet geen slappe hap worden.
      }

      \suggestion
      {
        Lekker met witte rijst.
      }

\end{recipe}
\label{rec:paksoi-met-walnoten}

\begin{recipe}
[ %
    preparationtime = {\unit[1]{h}},
    portion = {\portion{4 - 6}},
    bakingtime = {\unit[5]{m}},
    bakingtemperature={\protect\bakingtemperature{
        fanoven=\unit[220]{\textcelcius},
        }},
    calory,
    source = {Lekker gezond vegetarisch}
]
{Mexicaanse bonenschotel}

    \ingredients
    {%
        \unit[400]{g} & bruine bonen \\
        \unit[400]{g} & witte bonen \\
        \unit[400]{g} & kidneybonen \\
        \unit[1]{el} & olijfolie \\
        1 & rode ui \\
        1 & rode paprika \\
        \unit[2]{el} & tacokruiden \\
        1 blik & gepelde tomaten \\
        \unit[2]{dl} & gepureerde tomaten \\
        1 blikje & tomatenpuree \\
        1/2 zak & hete tacochips \\
        4 & jalope\~{n}opepertjes \\
        & zout en peper
    }

    \preparation
    {%
      \step Spoel alle bonen af en laat ze uitlekken. Snijd de ui in ringen.
            De paprika in repen. De gepelde tomaten in stukjes. Bewaar het sap.
            Snijd de pepertjes in ringetjes.
      \step Verhit de olijfolie in een pan en bak de ui en paprika 3 minuten.
      \step Roer de bonen, tacokruiden, gepelde tomaten, de gepureerde tomaten
            en de tomatenpuree erdoor. Voeg het tomatensap toe als het mengelsel
            te droog is. Breng op smaak met peper en zout
      \step Schep het bonenmengsel over in een ovenschaal. Verdeel er de nachochips
            over en bestrooi met jalope\~{n}opepertjes.
      \step Zet de schaal 5 minuten in een oven van 220 graden tot de nachochips
            warm zijn.
      \step Opdienen met rijst en bijvoorbeeld een komkommersalade.
    }

\end{recipe}
\label{rec:mexicaanse-bonensalade}


\section{Bijgerechten}
\begin{recipe}
[ %
    preparationtime = {\unit[20]{m}},
    portion = {\portion{6}},
    calory,
    source = {Gerbina van Hurk},
]
{Knoflook champignons}

    \ingredients
    {%
        \unit[450]{g} & champignons \\
         \unit[5]{el} & groene olijfolie \\
         2 teentjes & knoflook \\
         \unit[4]{el} & verse gehakte peterselie \\
        1 & citroen \\
         & zout en peper\\
    }

    \preparation
    {%
        \step Maak de champignons schoon en snijd de steeltjes er af.
        Snijd hele grote champignons in vieren.
        \step Verhit de olijfolie in een grote koekenpan en
        knijp hierin de knoflook uit met een knoflookpers.
        Laat het ongeveer 30 seconden fruiten.
        \step Voeg de champignons toe en zet het vuur hoog.
        Roer de champignons regelmatig om totdat de champignons
        alle olie uit de pan hebben opgezogen.
        \step Zet het vuur laag en wacht tot de champignons vocht vrij laten.
        Zet dan het vuur weer hoog en en bak ze gedurende 4-5 minuten
        al roerend tot al het vocht is verdampt.
        \step Snijd de citroen in 8 partjes. Knijp 1 partje uit boven
        de pan met champignons en breng het geheel op smaak met zout en peper.
        \step Roer de peterselie er door en bak deze kort mee om laag vuur.
        \step Schep de champignons in een bakje of
         schaaltje samen met het overgebleven bakvocht. Serveer de overige citroenpartjes er los bij.
      }

\end{recipe}
\label{rec:knoflook-champignons}


\section{Soepen}
\begin{recipe}
[ %
    preparationtime = {\unit[1]{h}},
    portion = {\portion{3 - 4}},
    calory,
    source = {Gerbina van Hurk},
]
{Knolselderijsoep}

    \ingredients
    {%
         \unit[20]{g} & boter \\
         \unit[1]{el} & olijfolie \\
         1 & knolselderij \\
         5 takjes & tijm   \\
         3 tenen  & knoflook \\
         1 blaadje & laurier \\
         \unit[1]{l} & groente boullion \\
         \unit[50] {g} & walnoten \\
         \unit[200]{ml} & kookroom \\
         & zout en peper\\
    }

    \preparation
    {%
        \step Schil de knolselderij en snijd hem in blokjes.
        Pel en plet de knoflook. Hak de walnoten.
        \step Smelt boter met olie in een grote pan, voeg knolselderij,
        takjes tijm en knoflook toe. Bak dit mengsel 10 minuten.
        \step Voeg laurier en bouillon toe, laat 20 minuten koken,
        tot de selderij zacht is. Verwijder de laurier.
        \step Rooster walnoten in een pan.
        \step Mix de soep glad in een blender. Roer de room door de soep.
        \step Bestrooi de soep met noten en tijmblaadjes.
      }

\end{recipe}
\label{rec:knolselderijsoep}


\section{Salades}
\begin{recipe}
[ %
    preparationtime = {\unit[20]{m}},
    portion = {\portion{4}},
    calory,
    source = {Gerbina van Hurk}
]
{Appel komkommer salade}

    \ingredients
    {%
         1 & komkommer \\
         2 & appels \\
         halve & citroen \\
         2 & bosuitjes \\
         $1/2$ bakje & verse dille \\
         \unit[4]{el} & yoghurt \\
         1 lepeltje & honing \\
         & peper\\
    }

    \preparation
    {%
        \step Appels wassen en in kleine stukjes snijden.
        \step Was de komkommer. Snijd hem in de lengte in vieren. Verwijder de
        zaadlijsten. En snijd ze in keine stukjes net zo groot als de appel stukjes.
        Meng de appel en de komkommer. Besprekel met een beetje citroen tegen het verkleuren
        van de appel.
        \step Snijd de bosuitjes in ringetjes en meng ze door de salade.
        \step Voor de dressing meng je de citroen, peper. Houd je van zoet doe er honing bij.
        Doe de yohurt bij de dressing en knip de verse dille er doorheen.
        \step Net voor het opdienen schep je de dressing door de salade.
      }

\end{recipe}
\label{rec:appel-komkommer-salade}

\begin{recipe}
[ %
    preparationtime = {\unit[10]{m}},
    portion = {\portion{4}},
    calory,
]
{Rucola met tomaat}

    \ingredients
    {%
         1 zakje & rucola \\
         1 grote of 2 kleine & tomaten \\
         1/4 & fijn gesnipperde ui \\
         & vinegrette \\
    }

    \preparation
    {%
        \step Snijd de gewone tomaten klein en voeg ze met de ui bij de rucola.
        \step Maak een vinegrette en serveer deze bij de salade.
      }

\end{recipe}

\begin{recipe}
[ %
    preparationtime = {\unit[\nicefrac{1}{2}]{h}},
    bakingtime,
    portion = {\portion{2}},
    calory,
    source = {Studentenkookboek, Berty van Essen}
]
{Roedjak}

    \ingredients
    {%
        \unit{1}{tl} & sambal \\
        \unit{1}{el} & bruine suiker \\
        \nicefrac{1}{2} & citroen \\
        \unit{\nicefrac{1}{2}}{el} & ketjap \\
         1 & handappel \\
         1 & handpeer \\
         \nicefrac{1}{2} & komkommer \\
         1 blikje & ananas \\
         & zout \\
    }

    \preparation
    {%
        \step Maak in een kom een sausje van de sambal, de suiker, het citroensap,
        de ketjap en zout naar smaak.
        \step Schil de appel en de peer, verwijder de klokhuizen en snijd ze in blokjes.
        Voeg de saus toe en schep het meteen door elkaar.
        \step Snijd de komkommer in blokjes, schep deze met uitgelekt blikje
        ananas door de rest van het fruit.
    }

\end{recipe}
\label{rec:roedjak}


\section{Sauzen}
\begin{recipe}
[ %
    preparationtime = {\unit[5]{m}},
    portion = {\portion{4 - 6}},
    calory,
]
{Vinegrette}

    \ingredients
    {%
         \unit[3]{el} & olijfolie \\
         \unit[2]{el} & mosterd \\
         \unit[1]{el} & azijn \\
         \unit[1]{el} & suiker of gembersiroop \\
         & zout en peper\\
    }

    \preparation
    {%
        \step Voeg de afgemeten ingredi\"{e}nten in een kommetje bijeen. En roer
        het goed.
        \step Voeg zout en peper naar smaak toe.
      }

\end{recipe}



\end{document}
