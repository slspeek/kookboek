\documentclass[%
a4paper,
%twoside,
12pt
]{book}

% encoding, font, language
\usepackage[T1]{fontenc}
\usepackage[latin1]{inputenc}
\usepackage{lmodern}
\usepackage[dutch]{babel}

\usepackage{nicefrac}

\usepackage[
    nowarnings,
    %myconfig
]
{xcookybooky}
% workaround for https://github.com/SvenHarder/xcookybooky/issues/13
\renewcommand{\step}
{%
    \stepcounter{step}%
    \lettrine
    [%
        lines=2,
        lhang=0,          % space into margin, value between 0 and 1
        loversize=0.15,   % enlarges the height of the capital
        slope=0em,
        findent=1em,      % gap between capital and intended text
        nindent=0em       % shifts all intended lines, begining with the second line
    ]{\thestep}{}%
}
% end workaround for https://github.com/SvenHarder/xcookybooky/issues/13

\DeclareRobustCommand{\textcelcius}{\ensuremath{^{\circ}\mathrm{C}}}

\setRecipeColors
{%
    recipename = black,
    ing = black,
    inghead = black,
    prep,
    prephead,
    hint,
    hinthead,
    numeration = black,
}

\setcounter{secnumdepth}{1}
\renewcommand*{\recipesection}[2][]
{%
    \subsection[#1]{#2}
}
\renewcommand{\subsectionmark}[1]
{% no implementation to display the section name instead
}

\usepackage{fancyhdr} 
% \fancyhf{}
\pagestyle{fancy}   
\fancyfoot{}
\fancyfoot[LE,RO]{\thepage}

%%%%%%%%%%
% hyperref
\usepackage{hyperref}    % must be the last package
\hypersetup{%
    pdfauthor            = {Steven Speek},
    pdftitle             = {Kookboek},
    pdfsubject           = {recepten},
    pdfstartview         = {FitV},
    pdfview              = {FitH},
    pdfpagemode          = {UseNone}, % Options; UseNone, UseOutlines
    bookmarksopen        = {true},
    pdfpagetransition    = {Glitter},
    colorlinks           = {true},
    linkcolor            = {blue},
    urlcolor             = {blue},
    citecolor            = {black},
    filecolor            = {black},
}
% hyperref
%%%%%%%%%%

\begin{document}

\setHeadlines
{% translation
    inghead = Ingredi\"{e}nten,
    prephead = Bereiding,
    hinthead = Tip,
    continuationhead = Voortzetting,
    continuationfoot = Gaat verder op volgende bladzijde,
    portionvalue = personen,
}

\title{Kookboek}
\author{Steven Speek}

\maketitle

\tableofcontents

\chapter{Hoofdgerechten}

\begin{recipe}
[ %
    preparationtime = {\unit[45]{min}},
    portion = {\portion{3 \`{a} 4}},
    source = {Budget koken}
]
{Andijvie rijstsalade}

    \ingredients
    {%
        \unit[350]{g} & andijvie \\
        \unit[200]{g} & rijst \\
        \unit[200]{g} & kapucijners \\
        \unit[200]{g} & ontbijtspek \\
        \unit[6]{el} & piccalilly \\
        \unit[2]{el} & olijfolie \\
        \unit[1]{tl} & poedersuiker \\
        \unit[1]{el} & azijn \\
        \unit[2]{teentjes} & knoflook \\
        & peper \\
        & zout
    }

    \preparation
    {%
        \step Kook de rijst gaar volgens de aanwijzingen op de verpakking. En laat haar afkoelen.
        \step Snijd het spek in kleine stukjes en bak het uit in ongeveer twintig minuten.
        \step Spoel de kapucijners af in een vergiet.
        \step Pers de knoflook uit in een kommetje voor de saus.
        \step Meng de piccalilly, olie, poedersuiker en azijn in het kommetje.
        \step Voeg het spekvet toe aan de saus.
        \step Meng in een grote pan: de kapucijners, andijvie, rijst en spek. Voeg de saus toe en meng het geheel goed. 
              Voeg peper en zout naar smaak toe.

    }
    
\end{recipe}

\begin{recipe}
[ %
    preparationtime = {\unit[\nicefrac{3}{4}]{h}},
    bakingtime = {\unit[20]{min} koken},
    portion = {\portion{2}},
    calory,
    source = {Rudolph's kitchen}
]
{Andijviestamppot met noten}

    \ingredients
    {%
        \unit[600]{g} & aardappelen\\
	\unit[500]{g} & fijngesneden andijvie \\
	\unit[1]{blokje} & groenteboullion \\
	\unit[150]{ml} & melk \\
	\unit[15]{g} & boter \\
	\unit[200]{g} & Goudse kaas in blokjes \\
	\unit[200]{g} & cashewnoten of pecannoten \\
    }

    \preparation
    {%
        \step Schil de aardappelen. Kook ze in water waar ze net onder staan met een boullionblokje aan de kook. Giet ze na 20 minuten af. En zet de pan heel even terug op het vuur om de aardappelen droog te stomen.
        \step Stamp de aardappelen fijn. Verwarm de melk en roer deze samen met het klontje boter door de aardappelen. 
	\step Roer de andijvie met een spatel door de aardappelpuree en de pan warm op heel laag vuur.
	\step Roer de kaasblokjes door het mengsel. Stop op tijd met roeren zodat de kaasblokjes nog zichtbaar blijven in de stamppot.
	\step Zet het deksel op de pan en houdt de stamppot enkele minuten warm zodat de kaas (gedeeltelijk) kan smelten. Rooster intussen de noten in een koekepan.
	\step Verdeel de stamppot over twee borden en strooi de noten eroverheen.
    }
    
\end{recipe}

\begin{recipe}
[ %
    preparationtime = {\unit[15]{m}},
    portion = {\portion{4}},
    calory,
    source = {Allerhande}
]
{Broccoli met sojasaus}

    \ingredients
    {%
         \unit[800]{g} & broccoli \\
         2 tenen & knoflook \\
         \unit[1]{el} & arachideolie \\
         \unit[1]{el} & sesamzaad \\
         \unit[3]{el} & sojasaus
    }

    \preparation
    {%
        \step Verdeel de broccoli in kleine roosjes.
              Schil de stronk en snijd in stukjes.
              Kook 3 minuten in een ruime pan water, giet af en laat goed uitlekken.
        \step Snijd ondertussen de knoflook in dunne plakjes.
              Verhit de olie in een wok en bak de knoflook en het sesamzaad
              1 minuut. Voeg de sojasaus en broccoli toe en bak 2 minuten
              al omscheppend op hoog vuur. Serveer direct.
      }

      \suggestion {%
        Lekker met (volkoren)rijst.
      }

\end{recipe}
%\label{recipe:broccoli-met-sojasaus}

\begin{recipe}
[ %
    preparationtime = {\unit[30]{min}},
    bakingtime = {\unit[40]{min} oventijd},
    bakingtemperature = {\protect\bakingtemperature{
	    fanoven=\unit[200]{\textcelcius}}},
    portion = {\portion{4}},
    calory,
    source = {Albert Heijn}
]
{Broccolitaart}

    \ingredients
    {%
	\unit[1]{pakje} & bladerdeeg\\
	\unit[1]{el} & boter\\
	\unit[1]{el} & paneermeel\\
	\unit[300]{g} & broccoli\\
	3 & eieren\\
	\unit[4]{el} & slagroom\\
	\unit[10]{sprietjes} & bieslook\\
	\unit[100]{g} & gezouten cashewnoten \\
	\unit[250]{g} & camembert\\
        & zout \\
        & peper \\
    }

    \preparation
    {%
	    \step Ontdooi de bladerdeeg.
	    \step Vet de springvorm in met boter.
	    \step Bekleed de springvorm met bladerdeeg. Zorg dat het goed aansluit. Bestrooi de bodem met paneermeel.
	    \step Was de broccoli. Snijd deze in roosjes en kleine plakjes. Kook deze ongeveer 5 minuten. Spoel ze onder koud water af en laat ze in een vergiet uitlekken.
	    \step Klop de eieren met de slagroom los in een kommetje. Voeg peper, zout en de bieslook fijngeknipt toe.
	    \step Kraak de cashnoten met de onderkant van een mok op de snijplank.
	    \step Snijd de camembert in plakken.
	    \step Verwarm de oven voor op 200\textcelcius
	    \step Rangschik de broccoli op de taartbodem en druk deze een beetje aan met een vork.
	    \step Strooi de gekraakte noten erover.
	    \step Leg de camembert erover.
	    \step Schenk het slagroommengsel erover.
	    \step Plaats de springvorm op een rooster onder het midden van de oven. Laat de taart gaar bakken in 40 minuten.
	    \step Neem de taart uit de oven en laat hem 10 minuten afkoelen. Verwijder voorzichtig de springvorm.

    }
    
\end{recipe}

\begin{recipe}
[ %
    preparationtime = {\unit[45]{min}},
    portion = {\portion{5}},
    source = {\href{https://www.ah.nl/allerhande/recept/R-R1192892/chili-con-carne-met-kikkererwten}{Allerhande}}
]
{Chili con carne}

    \introduction {
      Je hebt voor vijf personen als je er ook rijst bij maakt.
    }

    \ingredients
    {%
        1 & grote ui \\
        \unit[3]{tenen} & knoflook \\
        3 & puntpaprika's \\
        \unit[500]{g} & gehakt \\
        zak & \href{https://www.lidl.nl/p/chilischotel-mexicaanse-stijl/p6605666}{Chilischotel Mexicaanse stijl} \\
        blik & kidneybonen \\
        blik & zwarte bonen \\
        blik & doperwten \\
        blik & gepelde tomaten \\
        \unit[2\nicefrac{1}{2}]{el} & milde olijfolie \\
        \unit[1\nicefrac{1}{4}{tl} & chilipoeder \\
        \unit[2\nicefrac{1}{2}{tl} & gemalen komijnzaad \\
        \unit[1]{dl} & water \\
	         & peper \\
             & zout 
    }

    \preparation
    {%
	    \step Snijd de ui in halve ringen en de knoflook fijn. Maak de paprika's schoon en snijd in reepjes van 1 cm breed.
        \step Bak de gehakt rul in een aparte koekepan, voeg peper en zout toe.
	    \step Verhit de olie in een braadpan en fruit de ui en knoflook 5 min.
              op laag vuur. Voeg de chilipoeder en gemalen komijn toe en fruit 2 min. mee.
              Voeg de paprikareepjes toe en bak nog 5 min. op middelhoog vuur.
        \step Doe ondertussen de bonen en erwten in een vergiet en spoel onder koud stromend water.
              Laat goed uitlekken. Voeg het gehakt, de bonen, de erwten en tomaten en het water toe aan het paprikamengsel.
              Breng aan de kook, zet het vuur laag en laat de chili 15 min. met de deksel op de pan zachtjes stoven. 
    }

    \suggestion
    {
      Lekker met een beetje yoghurt
    }
    
\end{recipe}

\begin{recipe}
[ %
    preparationtime = {\unit[40]{min}},
    portion = {\portion{4}},
    source = {Albert Heijn}
]
{Maaltijdsoep met chorizo}

    \ingredients
    {%
	\unit[750]{ml} & water \\
	\unit[350]{g} & tomato frito\\
	\unit[200]{g} & chorizoworst\\
	\unit[1]{zak} & paprikamix\\
	\unit[150]{g} & uien (2 stuks)\\
	\unit[1]{tl} & gemalen komijnzaad (djinten) \\
	\unit[800]{g} & kikkererwten (2 blikken) \\ 
	\unit[230]{g} & doperwten (1 blik) \\
	1 & stokbrood \\
	    & zout \\
        & peper \\
    }

    \preparation
    {%
	    \step Breng het water aan de kook. Haal ondertussen het vel van de chorizo
		 en snijd de worst in blokjes van 1 cm. Verwijder de steelaanzet en zaadlijsten
		  van de paprika's en snijd het vruchtvlees in repen. Snijd de uien grof.
		\step Bak de chorizo 1 minuut in een soeppan zonder olie of boter. Voeg de paprika,
		 ui en gemalen komijn toe en bak al roerend 3 minuten op hoog vuur. 
		 Bak al roerend 3 minuten op hoog vuur. Laat ondertussen de kikkererwten en de doperwten 
		 uitlekken in een vergiet.
		\step Voeg de kikkererwten, tomato frito en het kokende water toe aan het chorizomengsel.
		Roer goed door, breng aan de kook, zet het vuur laag en laat 6 minuten koken. Breng op smaak met peper 
		en eventueel zout en verdeel de soep over de kommen.
    }

	\suggestion
    {
      Serveren met stokbrood
    }
    
\end{recipe}

\begin{recipe}
[ %
    preparationtime = {\unit[35]{min}},
    bakingtime = {\unit[30]{min} oventijd},
    bakingtemperature = {\protect\bakingtemperature{
	    fanoven=\unit[200]{\textcelcius}}},
    portion = {\portion{3}},
    calory,
    source = {Hazel Bathgate}
]
{Courgettetaart van de plaat}

    \ingredients
    {%
        \unit[200]{g} & geitenkaas zonder korst\\
	\unit[5]{el} & slagroom\\
	\unit[1]{tl} & komijnzaad\\
	1 & ei\\
	1 & courgette \\
	1 & (dunne) prei\\
	\unit[1]{teentje} & knoflook \\
	\unit[1]{pakje} & bladerdeeg\\
        bosje & munt \\
        & neutrale olie \\
        & zout \\
        & peper \\
    }

    \preparation
    {%
	    \step Verbrokkel de geitenkaas boven een kom. Pers de knoflook erdoor. Roer deze vervolgens los met de slagroom, de komijn, zout en peper. Het mengsel
	    is goed als de meeste klontjes zijn opgelost in de room.
	    \step Klop het ei los in een ander kommetje en de helft door het kaasmengel.
	    \step Schaaf de courgette met een kaasschaaf in lange plakken en snijd de prei in dunne ringen.
	    \step Meng in een kom de courgette met een snuf zout, peper en een scheutje olie.
	    \step Leg het bladerdeeg op de bakplaat. Houd een centimeter van de rand vrij en bestrijk het deeg (behalve de randen) met het geitenkaasmengsel.
	    \step Verwarm de oven voor op 200\textcelcius
	    \step Verdeel de prei over de plaattaart en leg de courgettelinten er losjes bovenop. Kruimel er vervolgens nog een paar stukje geitenkaas over.
	    \step Bestrijk de randen van de taart met ei en bestrooi het eventueel met zeezout.
	    \step Bak de plaattaart laag in de oven in ongeveer 30 minuten goudbruin en gaar.
	    \step Verscheur de munt en strooi de blaadjes vlak voor het serveren over de plaattaart.

    }
    
\end{recipe}

\begin{recipe}
[ %
    preparationtime = {\unit[20]{min}},
    portion = {\portion{2}}
]
{Gevulde pasta met bloemkoolrijst}

    \introduction {
      Je hebt 400 gram groente in totaal nodig.
    }
    \ingredients
    {%
        flesje & honingmosterd dressing \\
        verpakking & gevulde pasta \\
        zakje  & bloemkoolrijst \\
               & ijsbergsla/rucola \\
               & geraspte kaas \\
	%     & zout \\
    %     & peper \\
    }

    \preparation
    {%
	    \step Kook de gevulde pasta volgens de aanwijzingen op de verpakking.
	    \step Vul twee kleine schalen met de bloemkoolrijst en sla.
        \step Doe de dressing over de bloemkoolrijst
        \step Schep de gekookte pasta erop
        \step Strooi de geraspte kaas erop
    }
    
\end{recipe}

\begin{recipe}
[ %
    preparationtime = {\unit[\nicefrac{3}{4}]{h}},
    bakingtime = {\unit[20]{min} koken, \unit[10]{min} grillen},
    portion = {\portion{2}},
    calory,
    source = {Zusanova}
]
{Vegetarische hutspot}

    \ingredients
    {%
        \unit[500]{g} & aardappelen\\
        \unit[500]{g} & winterpeen of wortelen \\
        2 & grote uien \\
        \unit[1.5]{el} & gembersiroop \\
        \unit[0.5]{el} & sambal \\
        \unit[200]{g} & geitenkaas in plakken \\
        half bosje & koriander \\
        \unit[1]{el} & plantaardige olie \\
         & zout \\
    }

    \preparation
    {%
        \step Schil de aardappelen en de wortelen snijd ze in gelijke stukken.
        De uien in ringen snijden. Hak de koriander fijn.
        \step Kook de aardappelen met de wortel en zonder de ui in 20 minuten gaar. Doe er een beetje zout bij.
        \step Fruit de uiringen in de olie.
        \step Giet de aardappelen met de wortelen af. Stamp de ui, de sambal en gembersiroop door het aardappel mengsel.
        \step Schep dit mengsel in de een ovenschaal. Verdeel de plakken geitenkaas over de hutspot.
        Druppel nog wat gembersiroop op de geitenkaas.
        Grill dit net zolang tot de geitenkaas een licht bruine gloed krijgt.
        \step Serveer de schotel met de gehakte koriander.
    }
    
\end{recipe}

\begin{recipe}
[ %
    preparationtime = {\unit[30]{min}},
    bakingtime = {\unit[40 + 10]{min} oventijd},
    bakingtemperature = {\protect\bakingtemperature{
	     fanoven=\unit[200]{\textcelcius}}},
    portion = {\portion{2}},
    % calory,
    % source = {}
]
{Kawai}

    \ingredients
    {%
        1 & (kleine) ui \\
        1 & aubergine \\
        1 & paprika \\
        \unit[375]{gr} & aardappelen \\
        3 \'{a} 4 & tomaten  \\
        bosje & peterselie \\
        & olijflolie \\
        \unit[1/2]{el} & paprikapoeder \\
            & zout \\
            & peper \\
    }

    \preparation
    {%
	    \step Zet water op om de aardappelen in te koken.
	    \step Snijd de aardappelen in blokje niet groter dan 2 centimeter.
        \step Kook de aardappelblokjes 7 minuten.
        \step Snipper ondertussen de ui. Snijd de tomaten en paprika in blokjes. Knip de peterselie fijn.
        Snijd de aubergine in grote stukken.
        \step Doe de groenten met de aardappelblokjes in een ovenschaal. En voeg een scheut olijflolie toe. Voeg de helft van de peterselie toe.
        Voeg het paprikapoeder toe. Breng op smaak met peper en zout. Schep alles goed
        door elkaar.
        \step Dek af met aluminiumfolie en bak 40 minuten
        in de oven. 
        \step Haal de aluminiumfolie eraf en laat nog eens
        10 minuten doorbakken.
        \step Serveer direct uit de oven en
        bestrooi met de achtergehouden peterselie.
    }

    \suggestion
    {
        Lekker met brood of rijst (als je er met meer van wilt eten). 
    }
    
\end{recipe}

\begin{recipe}
[ %
    preparationtime = {\unit[45]{min}},
    portion = {\portion{2}},
    source = {\href{https://www.leukerecepten.nl/recepten/gezonde-pastasalade/}{Leuke recepten}}
]
{Maaltijdsalade courgette}

    \ingredients
    {%
        \unit[120]{gr}  & volkoren pasta \\
        1 & avocado \\
        Handje & rucola \\
        1 & courgette (of gegrilde paprika) \\
        \unit[1]{bol} & mozzarella \\
        & verse basilicum \\
        Handje & pistache  \\
        12 & cherrytomaatjes \\
        \unit[1/2]{el} & balsamico azijn \\
        \unit[1]{el} & citroensap \\
        \unit[1]{el} & olijfolie \\
        & peper \\
        & zout \\
    }

    \preparation
    {%
	    \step Kook de pasta gaar.
        \step Snijd de courgette in plakken met bijvoorbeeld kaasschaaf of
        mandoline en grill deze in een (grill) pan met een snufje peper en zout.
        \step  Meng het citroensap, de azijn en de olijfolie voor de dressing door elkaar.
        \step Snijd de tomaatjes doormidden. Halveer de avocado, verwijder de schil en pit en snijd het vruchtvlees in plakjes.
        \step Spoel de pasta af met
        koud water.
        \step Meng de pasta met de dressing, tomaatjes, gegrilde courgette en rucola. Scheur de
        mozzarella in stukjes en voeg toe aan de pastasalade. Voeg als laatste de plakjes
        avocado toe. Garneer de gezonde pastasalade met verse basilicum en pistachenootjes
    }
    
\end{recipe}

\begin{recipe}
[ %
preparationtime = {\unit[25]{min}},
portion = {\portion{2}},
source = {\href{https://uitpaulineskeuken.nl/recept/bonensalade-met-rode-ui-tomaat-en-balsamicodressing}{Paulines keuken}}
]
{Maaltijdsalade bonen met tomaat}


    \ingredients
    {%
        \unit[240]{g} & kidneybonen\\
        \unit[200]{g} & witte bonen\\
        \unit[1]{teentje} & knoflook \\
        \nicefrac{1}{3} zakje & rucolasla \\
        1 & rode ui \\
        \unit[200]{g} & gemengde cherrytomaatjes \\
        3 & tomaten \\
        half bosje & platte peterselie \\
        \unit[2\nicefrac{1}{2}]{el} & balsamico azijn \\
        \unit[2\nicefrac{1}{2}]{el} & olijfolie \\
        & volkoren stokbrood \\
        & echte boter \\
         & zout \\
         & peper \\
    }

    \preparation
    {%
        \step Snijd de knoflook fijn en de rode ui in dunne ringen. Meng de balsamicoazijn met de olijfolie en knoflook.
         Breng op smaak met zout en peper. 
        \step Giet het vocht uit de blikjes bonen en spoel ze goed af. Doe ze over in een pan voeg een scheutje water toe
         en wat zout en verwarm de bonen op een laag vuur. Giet af, doe terug in de pan en voeg direct de het balsamico mengsel toe.
         Schep goed om en laat met de deksel op de pan even staan. 
        \step Snijd de tomaten in plakken en halveer de cherrytomaten. Meng ze samen met de rode ui door de bonen.
         Schep alles over in een saladeschaal en garneer de bonensalade met grof gehakte peterselie. 
        \step Serveer de salade met het stokbrood.
    }
    
\end{recipe}

\begin{recipe}
[ %
    preparationtime = {\unit[45]{min}},
    portion = {\portion{3}}
]
{Nasi goreng}

    \ingredients
    {%
      \unit[3]{kopjes} & rijst \\ 
      \unit[1]{pak} & nasigroenten \\
      \unit[1]{bakje} & boemboe \\
      \unit[3]{teentjes} & knoflook \\
      \unit[1]{tl} & geraspte gember \\
      \unit[1]{tl} & sambal \\
      \unit[2]{el} & ketjap \\
      \unit[1]{zak} & kroepoek \\
      \unit[\nicefrac{1}{2}] & komkommer \\
      \unit[1]{bakje} & ketjap temp\'{e} \\
      \unit[2]{el} & olie \\
      \unit[3] & eieren
    }

    \preparation
    {%
	    \step Zet water op voor de rijst. Kook deze en laat haar staan.
      \step Begin verwijder de rode peper uit nasi pakket. Gebruik hem niet.
      \step Snijd de knoflook fijn en rasp de gember.
      \step Schil de komkommer en snijd hem in blokjes.
      \step Verhit de olie in een wok. Fruit de sambal 1 minuut los en fruit daarna de knoflook en gember mee.
      \step Voeg het nasipakket toe. Roerbak tot ze gaar is.
      \step Voeg de temp\'{e} toe.
      \step Schep de rijst erdoorheen.
      \step Bak de eieren als spiegelei in een aparte koekepan.
      \step Voeg de ketjap toe aan de nasi.
      \step Schep de komkommerblokjes door de nasi en roerbak haar kort mee.
      \step Serveer de nasi met kroepoek.
    }

    \suggestion
    {
      Hier past goed \hyperref[rec:pindasaus]{pindasaus} bij.
    }
    
\end{recipe}

\begin{recipe}
[ %
    preparationtime = {\unit[20]{m}},
    portion = {\portion{2}},
    calory,
    source = {MissRiz op Smulweb}
]
{Paksoi met walnoten}

    \ingredients
    {%
      1 struik & paksoi \\
      1 & vleestomaat \\
      \unit[1]{tl} & sambal \\
      2 teentjes & knoflook \\
      6 & walnoten \\
      1 & bouillonblokje \\
      snufje & djintan \\
      snufje & ketoembar\\
      \unit[1]{tl}  & sesamzaad
    }

    \preparation
    {%
        \step Paksoi in repen Paksoi in repen snijden,
              de stengels wat dunner dan de bladeren,
              knoflook fijn snijden,
              de tomaat in blokjes, walnoten kraken (niet hakken!)
        \step Knoflook 1 minuut fruiten, dan 2 minuten met de sambal erbij.
              tomaat erbij. Even bakken. paksoi erbij, bouillonblokje erboven
              verkruimelen, 2-3 minuten roerbakken. Vocht afgieten in zeef.
              Terug in de pan, en de kruiden erbij doen.
              Als laatste de walnoten erdoor scheppen.
              Niet te lang in de pan laten, want de paksoi moet geen slappe hap worden.
      }

      \suggestion
      {
        Lekker met witte rijst.
      }

\end{recipe}
\label{rec:paksoi-met-walnoten}

\begin{recipe}
[ %
    preparationtime = {\unit[20]{min}},
    portion = {\portion{2}},
    source = {\href{https://www.ah.nl/allerhande/recept/R-R1199964/pasta-pesto-met-kip}{Allerhande}}
]
{Pasta pesto met kip}

    \ingredients
      {%
      \unit[150]{g} & penne \\
      \unit[250]{g} & snoeptomaatjes \\
      \unit[50]{g} & groene pesto \\
      \unit[50]{g} & rucola \\
      \unit[100]{g} & kipreepjes \\
      \unit[1]{el} & milde olijfolie \\
      & peper \\
      & zout \\
      }

    \preparation
      {%
        \step Kook de pasta volgens de aanwijzingen op de verpakking en giet af.
        \step Halveer ondertussen de tomaten. Meng de pasta met de pesto, rucola, tomaat, olijolie en kipreepjes.
        Breng op smaak met peper en zout.
      }
    
\end{recipe}

\begin{recipe}
[ %
    preparationtime = {\unit[30]{min}},
    bakingtime = {\unit[40]{min} bakken},
	portion = {\portion{3 \`{a} 4}},
    calory,
    source = {\href{https://uitpaulineskeuken.nl/recept/spaanse-tortilla-met-aardappel}{Paulines keuken}}
]
{Spaanse tortilla met aardappel}

    \ingredients
    {%
        \unit[500]{g} & aardappelen\\
	\unit[1]{teentje} & knoflook\\
	\unit[75]{g} & geraspte kaas\\
        & peper en zout \\
        1 & ui \\
	1 & paprika\\
	5 & eieren\\
	handje & peterselie\\
	& olijfolie\\
    }

    \preparation
    {%
	    \step Schil de aardappelen en snijd ze in plakjes. Kook ze 3 minuten voor en laat ze in vergiet uitlekken (en nagaren).
	    \step Snijd de ui in halve ringen en hak de knoflook fijn en fruit dit aan in een ruime koekenpan.
	    \step Snijd de paprika in blokjes en bak deze tien minuten mee. Voeg de aardappelschijfjes toe en zorg dat alles goed mengt.
	    \step Klop de eieren luchtig en voeg peper en zout toe. Voeg de peterselie fijngeknipt toe. Voeg ook de geraspte kaas toe.
	    	Giet het eiermengsel in de pan met de aardappelen. Laat de tortilla nu 20 minuten zachtjes garen op de (vrijwel) laagste gasstand.
	    \step Draai de tortilla voorzichtig om met behulp van een bord. Bak de andere kant vervolgens ook nog even 10 minuten aan.
	    \step Serveer in punten.
    }
    
\end{recipe}

\begin{recipe}
[ %
    preparationtime = {\unit[\nicefrac{1}{2}]{h}},
    portion = {\portion{2}},
    calory,
]
{Tabbouleh}

    \introduction {
      Dit gerecht vraagt vanwege de avocado planning. Deze moet enkele dagen
      tot een week voor het maken worden aangeschaft.
    }
    \ingredients
    {%
        \unit[150]{ml} & couscous \\
        1 & komkommer \\
        2 & tomaten \\
        1 & avocado \\
        1 zakje & gesneden rauwkost \\
        1 bos & koriander \\
        2 tenen & knoflook \\
        \nicefrac{1}{2} beker & yoghurt\\
        \nicefrac{1}{2} & citroen\\
        & peper en zout \\
        & olijfolie \\
        zakje & shoarmakruiden;
    }

    \preparation
    {%
        \step Doe een scheut olijfolie bij de de couscous. Wel dit met 150 ml kokend water en laat haar vijf minuten staan.
        \step Snijd de tomaten en de helft van de komkommer in blokjes. Knip koriander fijn.
         Snijd de avocado klein. Meng dit alles door de couscous met peper en zout.
        \step Voor het sausje rasp je de overgebleven komkommer.
        Knijp de 2 tenen knoflook erdoorheen, yoghurt en citroen erbij en
        op smaak brengen met peper en zout.
        \step Serveer de tabbouleh, de komkommer-saus en de shoarmakruiden los van elkaar.
      }

\end{recipe}

\begin{recipe}
[ %
    preparationtime = {\unit[1]{h}},
    bakingtime,
    portion = {\portion{2}},
    calory,
    source = {Studentenkookboek, Berty van Essen}
]
{Tahoe met ketjap}

    \ingredients
    {%
        klein blok & tahoe (biologisch) \\
        \unit[200]{g} & zilvervlies rijst\\
        $$1/2$$ & citroen \\
        \unit{2}{el} & olie \\
        een kleine  &  ui \\
        3 teentjes  & knoflook \\
        \unit{2}{tl} & sambal \\
        \unit{3}{el} & ketjap \\
        \unit{200}{g}  & taug\'{e} \\
        & zout \\
    }

    \preparation
    {%
        \step Snijd de tahoe in blokjes of balkjes van 1 cm. Wrijf ze in met wat zout.
        Sprenkel de citroen erover en laat dit minstens 10 minuten staan.
        \step Snipper de ui en de knoflook.
        \step Verhit de olie in een koekepan en bak de tahoe snel bruin. Neem ze uit de
        pan en laat ze in een vergiet uitlekken.
        \step Bak in de zelfde olie de ui en knoflook glazig en goudgeel. Voeg
        de sambal, ketjap en zout toe. Was de taug\'{e} en bak dat even mee. Voeg eventueel
        wat water toe als de saus te dik wordt. Voeg de tahoe weer toe en je kunt opdienen.
    }

    \suggestion
    {
      Hier past goed Roedjak bij. Met iets zoets als toetje is het dan een hele maaltijd.
    }
\end{recipe}

\begin{recipe}
[ %
    preparationtime = {\unit[45]{min}},
    portion = {\portion{3}},
    source = {\href{https://www.leukerecepten.nl/recepten/zoete-aardappel-stamppot/}{Leuke recepten}}
]
{Zoete aardappel spinazie stampot}

    \ingredients
    {%
    \unit[1]{kg} & zoete aardappels \\
    \unit[200]{g} & verse spinazie \\
    klontje & boter of margarine \\
    \unit[100]{g} & zachte geitenkaas \\
    handje & walnoten \\
    6 & chipolata worstjes \\
    \unit[\nicefrac{1}{2}]{tl} & kurkuma \\
    & cayennepeper \\ 
    & zout \\
    & peper \\
    }

    \preparation
    {%
      \step Zet een grote pan water op met water.
      \step Breek de halve walnoten in vieren.
      \step Schil de zoete aardappelen. En snijd ze in grote stukken. Kook ze in 15 minuten gaar.
      \step Braad ondertussen de worstjes in een koekenpan met een beetje boter of margarine.
      \step Giet de zoete aardappel af en vang wat van het kookvocht op. Stamp de zoete aardappels
            fijn met een stamper en voeg een beetje boter of margarine toe en beetje bij beetje wat
            van het kookvocht tot de puree smeu\"{i}g is. Breng op smaak met peper en zout.
      \step Schep de spinazie erdoor tot deze iets begint te slinken.
      \step Snijd de plakjes geitenkaas in vieren en voeg ze met de walnoten toe. En roer ze door de stampot. 
      \step Serveer de stampot met de worstjes.
    }
    
\end{recipe}


\chapter{Bijgerechten}

\begin{recipe}
[ %
    preparationtime = {\unit[1/4]{h}},
    bakingtime = {\unit[3]{min} koken, \unit[10]{min} bakken},
    portion = {\portion{1}},
    calory,
    source = {Steven Speek}
]
{Gebakken aardappelen}

    \ingredients
    {%
        \unit[200]{g} & aardappelen\\
	& peper en zout \\
	& olie \\
    }

    \preparation
    {%
        \step Schil de aardappel(en).
        \step Snijd de aardappelen in plakken van nog geen halve centimeter dik.
        \step Kook ze plakken 3 minuten.
        \step Giet de aardappelen af.
	\step Bak de aardappelen 10 minuten in de olie.
    }
    
\end{recipe}

\begin{recipe}
[ %
    preparationtime = {\unit[20]{m}},
    portion = {\portion{6}},
    calory,
    source = {Gerbina van Hurk},
]
{Knoflook champignons}

    \ingredients
    {%
        \unit[450]{g} & champignons \\
         \unit[5]{el} & groene olijfolie \\
         2 teentjes & knoflook \\
         \unit[4]{el} & verse gehakte peterselie \\
        1 & citroen \\
         & zout en peper\\
    }

    \preparation
    {%
        \step Maak de champignons schoon en snijd de steeltjes er af.
        Snijd hele grote champignons in vieren.
        \step Verhit de olijfolie in een grote koekenpan en
        knijp hierin de knoflook uit met een knoflookpers.
        Laat het ongeveer 30 seconden fruiten.
        \step Voeg de champignons toe en zet het vuur hoog.
        Roer de champignons regelmatig om totdat de champignons
        alle olie uit de pan hebben opgezogen.
        \step Zet het vuur laag en wacht tot de champignons vocht vrij laten.
        Zet dan het vuur weer hoog en en bak ze gedurende 4-5 minuten
        al roerend tot al het vocht is verdampt.
        \step Snijd de citroen in 8 partjes. Knijp 1 partje uit boven
        de pan met champignons en breng het geheel op smaak met zout en peper.
        \step Roer de peterselie er door en bak deze kort mee om laag vuur.
        \step Schep de champignons in een bakje of
         schaaltje samen met het overgebleven bakvocht. Serveer de overige citroenpartjes er los bij.
      }

\end{recipe}
\label{rec:knoflook-champignons}


\chapter{Soepen}

\begin{recipe}
[ %
    preparationtime = {\unit[1]{h}},
    portion = {\portion{3 - 4}},
    calory,
    source = {Gerbina van Hurk},
]
{Knolselderijsoep}

    \ingredients
    {%
         \unit[20]{g} & boter \\
         \unit[1]{el} & olijfolie \\
         1 & knolselderij \\
         5 takjes & tijm   \\
         3 tenen  & knoflook \\
         1 blaadje & laurier \\
         \unit[1]{l} & groente boullion \\
         \unit[50] {g} & walnoten \\
         \unit[200]{ml} & kookroom \\
         & zout en peper\\
    }

    \preparation
    {%
        \step Schil de knolselderij en snijd hem in blokjes.
        Pel en plet de knoflook. Hak de walnoten.
        \step Smelt boter met olie in een grote pan, voeg knolselderij,
        takjes tijm en knoflook toe. Bak dit mengsel 10 minuten.
        \step Voeg laurier en bouillon toe, laat 20 minuten koken,
        tot de selderij zacht is. Verwijder de laurier.
        \step Rooster walnoten in een pan.
        \step Mix de soep glad in een blender. Roer de room door de soep.
        \step Bestrooi de soep met noten en tijmblaadjes.
      }

\end{recipe}
\label{rec:knolselderijsoep}


\chapter{Salades}

\begin{recipe}
[ %
    preparationtime = {\unit[20]{minuten}},
    calory,
    source = {Hazel Bathgate}
]
{Bietensalade met geitenkaas}

    \ingredients
    {%
	    \unit[1]{verpakking} & biologische bieten (LIDL)\\
	    \unit[200]{g} & zachte geitenkaas (2 rollen)\\
	    \unit[100]{g} & walnoten\\
	    \unit[1]{teentje} & knoflook\\
	    beetje & olijfolie\\
	    & peper en zout\\
    }

    \preparation
    {%
	\step Snijd de bieten in kleine stukjes. En doe dit in een kom.
	\step Snijd gietenkaas in kleine stukjes en voeg dit aan de bieten toe.
	\step Pers het teentje knoflook erdoor.
	\step Kraak de walnoten en voeg ze toe.
	\step Voeg een scheutje olijfolie toe.
	\step Meng dit alles goed en voeg peper en zout naar smaak toe.
	
    }

\end{recipe}

\begin{recipe}
[ %
    preparationtime = {\unit[20]{m}},
    portion = {\portion{4}},
    calory,
    source = {Gerbina van Hurk}
]
{Appel komkommer salade}

    \ingredients
    {%
         1 & komkommer \\
         2 & appels \\
         halve & citroen \\
         2 & bosuitjes \\
         $1/2$ bakje & verse dille \\
         \unit[4]{el} & yoghurt \\
         1 lepeltje & honing \\
         & peper\\
    }

    \preparation
    {%
        \step Appels wassen en in kleine stukjes snijden.
        \step Was de komkommer. Snijd hem in de lengte in vieren. Verwijder de
        zaadlijsten. En snijd ze in keine stukjes net zo groot als de appel stukjes.
        Meng de appel en de komkommer. Besprekel met een beetje citroen tegen het verkleuren
        van de appel.
        \step Snijd de bosuitjes in ringetjes en meng ze door de salade.
        \step Voor de dressing meng je de citroen, peper. Houd je van zoet doe er honing bij.
        Doe de yohurt bij de dressing en knip de verse dille er doorheen.
        \step Net voor het opdienen schep je de dressing door de salade.
      }

\end{recipe}
\label{rec:appel-komkommer-salade}

\begin{recipe}
[ %
    preparationtime = {\unit[10]{m}},
    portion = {\portion{4}},
    calory,
]
{Rucola met tomaat}

    \ingredients
    {%
         1 zakje & rucola \\
         1 grote of 2 kleine & tomaten \\
         1/4 & fijn gesnipperde ui \\
         & vinegrette \\
    }

    \preparation
    {%
        \step Snijd de gewone tomaten klein en voeg ze met de ui bij de rucola.
        \step Maak een vinegrette en serveer deze bij de salade.
      }

\end{recipe}

\begin{recipe}
[ %
    preparationtime = {\unit[\nicefrac{1}{2}]{h}},
    bakingtime,
    portion = {\portion{2}},
    calory,
    source = {Studentenkookboek, Berty van Essen}
]
{Roedjak}

    \ingredients
    {%
        \unit{1}{tl} & sambal \\
        \unit{1}{el} & bruine suiker \\
        \nicefrac{1}{2} & citroen \\
        \unit{\nicefrac{1}{2}}{el} & ketjap \\
         1 & handappel \\
         1 & handpeer \\
         \nicefrac{1}{2} & komkommer \\
         1 blikje & ananas \\
         & zout \\
    }

    \preparation
    {%
        \step Maak in een kom een sausje van de sambal, de suiker, het citroensap,
        de ketjap en zout naar smaak.
        \step Schil de appel en de peer, verwijder de klokhuizen en snijd ze in blokjes.
        Voeg de saus toe en schep het meteen door elkaar.
        \step Snijd de komkommer in blokjes, schep deze met uitgelekt blikje
        ananas door de rest van het fruit.
    }

\end{recipe}
\label{rec:roedjak}


\chapter{Sauzen}

\begin{recipe}
[ %
    preparationtime = {\unit[15]{min}},
    portion = {\portion{3}},
    % source = {\href{https://www.leukerecepten.nl/recepten/gezonde-pastasalade/}{Leuke recepten}}
]
{Pindasaus}

    \ingredients
    {%
      \unit[\nicefrac{1}{2}]{tl} & sambal \\
      \unit[5]{el} & pindakaas \\
      \unit[1]{tl} & geraspte gember \\
      \unit[100]{ml} & water \\
      \unit[2]{el} & ketjap
    }

    \preparation
    {%
	    \step Doe de ingredi\"{e}nten in een sauspannetje.
	    \step Breng al roerend met een garde aan de kook.
      \step Laat 3 minuten zachtjes koken.
    }
    
\end{recipe}
\label{rec:pindasaus}

\begin{recipe}
[ %
    preparationtime = {\unit[5]{m}},
    portion = {\portion{4 - 6}},
    calory,
]
{Vinegrette}

    \ingredients
    {%
         \unit[3]{el} & olijfolie \\
         \unit[2]{el} & mosterd \\
         \unit[1]{el} & azijn \\
         \unit[1]{el} & suiker of gembersiroop \\
         & zout en peper\\
    }

    \preparation
    {%
        \step Voeg de afgemeten ingredi\"{e}nten in een kommetje bijeen. En roer
        het goed.
        \step Voeg zout en peper naar smaak toe.
      }

\end{recipe}


\appendix

\chapter{Maten en gewichten}

\begin{tabular}{|c|c|}
    \hline
    \unit[1]{kopje} & \unit[125]{ml} \\
    \unit[1]{beker} & \unit[180]{ml} \\
    1 ui &  \unit[75]{g} \\
    1 tomaat & \unit[70]{g} \\
    \hline
\end{tabular}

\end{document}
